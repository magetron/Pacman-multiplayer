\PassOptionsToPackage{unicode=true}{hyperref} % options for packages loaded elsewhere
\PassOptionsToPackage{hyphens}{url}
%
\documentclass[]{article}
\usepackage{lmodern}
\usepackage{amssymb,amsmath}
\usepackage{ifxetex,ifluatex}
\usepackage{fixltx2e} % provides \textsubscript
\ifnum 0\ifxetex 1\fi\ifluatex 1\fi=0 % if pdftex
  \usepackage[T1]{fontenc}
  \usepackage[utf8]{inputenc}
  \usepackage{textcomp} % provides euro and other symbols
\else % if luatex or xelatex
  \usepackage{unicode-math}
  \defaultfontfeatures{Ligatures=TeX,Scale=MatchLowercase}
\fi
% use upquote if available, for straight quotes in verbatim environments
\IfFileExists{upquote.sty}{\usepackage{upquote}}{}
% use microtype if available
\IfFileExists{microtype.sty}{%
\usepackage[]{microtype}
\UseMicrotypeSet[protrusion]{basicmath} % disable protrusion for tt fonts
}{}
\IfFileExists{parskip.sty}{%
\usepackage{parskip}
}{% else
\setlength{\parindent}{0pt}
\setlength{\parskip}{6pt plus 2pt minus 1pt}
}
\usepackage{hyperref}
\hypersetup{
            pdfborder={0 0 0},
            breaklinks=true}
\urlstyle{same}  % don't use monospace font for urls
\usepackage{longtable,booktabs}
% Fix footnotes in tables (requires footnote package)
\IfFileExists{footnote.sty}{\usepackage{footnote}\makesavenoteenv{longtable}}{}
\setlength{\emergencystretch}{3em}  % prevent overfull lines
\providecommand{\tightlist}{%
  \setlength{\itemsep}{0pt}\setlength{\parskip}{0pt}}
\setcounter{secnumdepth}{0}
% Redefines (sub)paragraphs to behave more like sections
\ifx\paragraph\undefined\else
\let\oldparagraph\paragraph
\renewcommand{\paragraph}[1]{\oldparagraph{#1}\mbox{}}
\fi
\ifx\subparagraph\undefined\else
\let\oldsubparagraph\subparagraph
\renewcommand{\subparagraph}[1]{\oldsubparagraph{#1}\mbox{}}
\fi

% set default figure placement to htbp
\makeatletter
\def\fps@figure{htbp}
\makeatother


\date{}

\begin{document}

\hypertarget{pacman-network-specification}{%
\section{Pacman Network
Specification}\label{pacman-network-specification}}

Author : Daiqi Wu

Email : \url{daiqi.wu.17@ucl.ac.uk}

Student Number : 17134402

\hypertarget{contents}{%
\subsection{Contents}\label{contents}}

\begin{itemize}
\tightlist
\item
  Terminology
\item
  General Description
\item
  Message Types
\item
  Message Contents and Encoding
\item
  Message Sanity Check
\item
  Message Timing
\end{itemize}

\hypertarget{terminology}{%
\subsection{Terminology}\label{terminology}}

This specification uses the terms MUST, SHOULD, and MAY as defined in
RFC 2119 {[}rfc2119{]}.

This specification also uses the terms LOCAL, AWAY, REMOTE and FOREIGN
as defined in assigment PDF of ENGF0002.

This protocol is generally a PEER-TO-PEER Network Model as definied in
\emph{Distributed Application Architecture} {[}Sun Microsystem, 2011{]}.

We also define addtional terms to specifiy the computer the software
instance is running on :

\begin{itemize}
\item
  LOCAL\_CLIENT : The LOCAL game instance that is running on localhost.
\item
  REMOTE\_CLIENT : The REMOTE instance that is connected to the LOCAL
  game instance.
\end{itemize}

The Pacman protocol runs over TCP and UDP depending on message types,
using port 27006.

There are 12 types of messages :

\begin{itemize}
\tightlist
\item
  \texttt{maze\_update}
\item
  \texttt{pacman\_arrived}
\item
  \texttt{pacman\_left}
\item
  \texttt{pacman\_died}
\item
  \texttt{pacman\_go\_home}
\item
  \texttt{foreign\_pacman\_left}
\item
  \texttt{pacman\_ate\_ghost}
\item
  \texttt{pacman\_update}
\item
  \texttt{ghost\_update}
\item
  \texttt{eat}
\item
  \texttt{score\_update}
\item
  \texttt{status\_update}
\end{itemize}

\hypertarget{general-description}{%
\subsection{General Description}\label{general-description}}

At game start, \texttt{maze\_update} message is exchanged between
LOCAL\_CLIENT and REMOTE\_CLIENT such that both clients keeps an update
of each other's maze.

When LOCAL pacman is in the LOCAL maze, LOCAL\_CLIENT sends \texttt{eat}
messages to REMOTE\_CLIENT such that REMOTE\_CLIENT updates their REMOTE
maze, and vice versa.

When LOCAL pacman is AWAY, a \texttt{pacman\_arrived} message is sent to
REMOTE\_CLIENT. Afterwards, \texttt{pacman\_update} will be sent to
REMOTE\_CLIENT to display our AWAY pacman.

Since CLIENTs always keep an updated copy of REMOTE maze, the eating
done by LOCAL pacman is calculated by LOCAL\_CLIENT and updates
REMOTE\_CLIENT via \texttt{eat} messages at all times.

When LOCAL pacman is AWAY, \texttt{ghost\_update} message will be
received from REMOTE\_CLIENT as it detects a FOREIGN pacman. Then,
LOCAL\_CLIENT is responsilbe for calculating REMOTE ghosts and their
interactions with our AWAY pacman.

If a REMOTE ghost is eaten by AWAY pacman, \texttt{pacman\_ate\_ghost}
is sent to update REMOTE\_CLIENT. If a AWAY pacman is eaten by a REMOTE
ghost, LOCAL\_CLIENT will update REMOTE\_CLIENT with
\texttt{pacman\_died}.

When LOCAL pacman returns from REMOTE maze, LOCAL\_CLIENT will send a
\texttt{pacman\_left} to inform REMOTE\_CLIENT of stop displaying the
FOREIGN pacman.

When a level fails (e.g.~one of the player runs out of lives) or the
game progresses to the next level, REMOTE\_CLIENT sends a
\texttt{pacman\_go\_home} message to LOCAL\_CLIENT to forcibly return
their FOREIGN pacman. LOCAL\_CLIENT will confirm this leave by replying
a \texttt{foreign\_pacman\_left.}

When the score of the LOCAL pacman changes, LOCAL\_CLIENT shall update
the REMOTE\_CLIENT by sending a \texttt{score\_update} message.

In the following scenario when :

\begin{itemize}
\tightlist
\item
  players fail a level,
\item
  game progresses to a next level,
\item
  game gets initialised,
\item
  the state of ghosts change,
\end{itemize}

a \texttt{status\_update} message will be sent from the LOCAL\_CLIENT to
inform the REMOTE\_CLIENT of its gamestate and vice versa.

\hypertarget{message-types}{%
\subsection{Message Types}\label{message-types}}

Among the above 12 types of messages,

\begin{itemize}
\tightlist
\item
  \texttt{maze\_update},
\item
  \texttt{status\_update},
\item
  \texttt{pacman\_arrived},
\item
  \texttt{pacman\_left},
\item
  \texttt{pacman\_died},
\item
  \texttt{pacman\_go\_home},
\item
  \texttt{foreign\_pacman\_left},
\item
  \texttt{pacman\_ate\_ghost}
\end{itemize}

are sent using TCP while

\begin{itemize}
\tightlist
\item
  \texttt{pacman\_update},
\item
  \texttt{ghost\_update},
\item
  \texttt{eat},
\item
  \texttt{score\_update}
\end{itemize}

are sent using UDP.

The reason for this is that \texttt{maze\_update} is called less common
and sends large amount of data at once. Other game state messages sent
through TCP are generally less commonily called every round.

The messages sent using UDP are usually update messages which requires a
more timely delivery of information. Since game state updates every 20
ms, it is acceptable to lose few packages since newer packages will make
up the loss.

\hypertarget{message-contents-and-encoding}{%
\subsection{Message Contents and
Encoding}\label{message-contents-and-encoding}}

\hypertarget{messages-transmitted-using-tcp}{%
\subsubsection{Messages transmitted using
TCP}\label{messages-transmitted-using-tcp}}

\hypertarget{maze_update}{%
\paragraph{\texorpdfstring{\texttt{maze\_update}}{maze\_update}}\label{maze_update}}

\texttt{maze\_update} message consists of a 166 byte package as follows
:

\begin{longtable}[]{@{}lllll@{}}
\toprule
0 - 3 & 4 - 1253 & 1254 - 1303 & 1304 - 1323 & 1324 -
1327\tabularnewline
\midrule
\endhead
M\_TYPE & MAZE\_MATRIX & PILL\_LOC & END\_LOC & UNUSED\tabularnewline
\bottomrule
\end{longtable}

\begin{itemize}
\item
  M\_TYPE : a 4 bit type field that specifies message types.
  \texttt{maze\_update} has decimal value 0.
\item
  MAZE\_MATRIX : a 1250 bit field that consists of a 25 * 25 matrix with
  2 bits per element. Each element have a decimal value of 0, 1, 2 which
  indicates wall, road with food and road without food.
\item
  PILL\_LOC : a 50 bit field that consists of 5 pairs of coordinates X
  Y. The coordinate value X Y are both 5 bit unsigned integer describing
  the location of power pills.
\item
  END\_LOC : a 20 bit field that consists of 2 pairs of coordinates X Y
  with same data type as PILL\_LOC that desribes the end point of the
  tunnel A and B.
\item
  UNUSED : a 4 bit field used for bit alignment.
\end{itemize}

\hypertarget{status_update}{%
\paragraph{\texorpdfstring{\texttt{status\_update}}{status\_update}}\label{status_update}}

\texttt{status\_update} message consists of a 1 byte package as follows
:

\begin{longtable}[]{@{}lll@{}}
\toprule
0 - 3 & 4 - 6 & 7\tabularnewline
\midrule
\endhead
M\_TYPE & G\_STATUS & UNUSED\tabularnewline
\bottomrule
\end{longtable}

\begin{itemize}
\item
  M\_TYPE : a 4 bit type field that specifies message types.
  \texttt{status\_update} has decimal value 1.
\item
  G\_STATUS : a 3 bit unsigned integer that specifies the updating
  gamestate.
\end{itemize}

\begin{quote}
\begin{longtable}[]{@{}ll@{}}
\toprule
Decimal Value & Gamestate\tabularnewline
\midrule
\endhead
0 & STARTUP\tabularnewline
1 & CHASE\tabularnewline
2 & FRIGHTEN\tabularnewline
3 & GAME\_OVER\tabularnewline
4 & NEXT\_LEVEL\_WAIT\tabularnewline
5 & READY\_TO\_RESTART\tabularnewline
6 & ILLEGAL\tabularnewline
7 & ILLEGAL\tabularnewline
\bottomrule
\end{longtable}

Note : If received G\_STATUS of 6 or 7, CLIENT should send
\texttt{status\_update} with G\_STATUS of 0 to reinitialise the game.
\end{quote}

\begin{itemize}
\tightlist
\item
  UNUSED : a 1 bit field used for bit alignment
\end{itemize}

\hypertarget{pacman_arrived}{%
\paragraph{\texorpdfstring{\texttt{pacman\_arrived}}{pacman\_arrived}}\label{pacman_arrived}}

\texttt{pacman\_arrived} message consists a byte package as follows :

\begin{longtable}[]{@{}ll@{}}
\toprule
0 - 3 & 4 - 7\tabularnewline
\midrule
\endhead
M\_TYPE & UNUSED\tabularnewline
\bottomrule
\end{longtable}

\begin{itemize}
\item
  M\_TYPE : a 4 bit type field that specifies message types.
  \texttt{pacman\_arrived} has decimal value 2.
\item
  UNUSED : a 4 bit field used for bit alignment
\end{itemize}

\hypertarget{pacman_left}{%
\paragraph{\texorpdfstring{\texttt{pacman\_left}}{pacman\_left}}\label{pacman_left}}

\texttt{pacman\_left} message consists of a byte package as follows :

\begin{longtable}[]{@{}ll@{}}
\toprule
0 - 3 & 4 - 7\tabularnewline
\midrule
\endhead
M\_TYPE & UNUSED\tabularnewline
\bottomrule
\end{longtable}

\begin{itemize}
\item
  M\_TYPE : a 4 bit type field that specifies message types.
  \texttt{pacman\_left} has decimal value 3.
\item
  UNUSED : a 4 bit field used for bit alignment
\end{itemize}

\hypertarget{pacman_died}{%
\paragraph{\texorpdfstring{\texttt{pacman\_died}}{pacman\_died}}\label{pacman_died}}

\texttt{pacman\_died} message consists of a 4 byte package as follows :

\begin{longtable}[]{@{}lll@{}}
\toprule
0 - 3 & 4 - 13 & 14 - 15\tabularnewline
\midrule
\endhead
M\_TYPE & DIE\_LOC & UNUSED\tabularnewline
\bottomrule
\end{longtable}

\begin{itemize}
\item
  M\_TYPE : a 4 bit type field that specifies message types.
  \texttt{pacman\_died} has decimal value 4.
\item
  DIE\_LOC : a pair of coordinate X Y with two unsigned 5 bit integer to
  describe the death location of the AWAY pacman on the REMOTE maze.
\item
  UNUSED : a 2 bit field used for bit alignment.
\end{itemize}

\hypertarget{pacman_go_home}{%
\paragraph{\texorpdfstring{\texttt{pacman\_go\_home}}{pacman\_go\_home}}\label{pacman_go_home}}

\texttt{pacman\_go\_home} message consists of a byte package as follows
:

\begin{longtable}[]{@{}ll@{}}
\toprule
0 - 3 & 4 - 7\tabularnewline
\midrule
\endhead
M\_TYPE & UNUSED\tabularnewline
\bottomrule
\end{longtable}

\begin{itemize}
\item
  M\_TYPE : a 4 bit type field that specifies message types.
  \texttt{pacman\_go\_home} has decimal value 5.
\item
  UNUSED : a 4 bit field used for bit alignment.
\end{itemize}

\hypertarget{foreign_pacman_left}{%
\paragraph{\texorpdfstring{\texttt{foreign\_pacman\_left}}{foreign\_pacman\_left}}\label{foreign_pacman_left}}

\texttt{foreign\_pacman\_left} message consists of a byte package as
follows :

\begin{longtable}[]{@{}ll@{}}
\toprule
0 - 3 & 4 - 7\tabularnewline
\midrule
\endhead
M\_TYPE & UNUSED\tabularnewline
\bottomrule
\end{longtable}

\begin{itemize}
\item
  M\_TYPE : a 4 bit type field that specifies message types.
  \texttt{foreign\_pacman\_left} has decimal value 6.
\item
  UNUSED : a 4 bit field used for bit alignment.
\end{itemize}

\hypertarget{pacman_ate_ghost}{%
\paragraph{\texorpdfstring{\texttt{pacman\_ate\_ghost}}{pacman\_ate\_ghost}}\label{pacman_ate_ghost}}

\texttt{pacman\_ate\_ghost} message consists of a byte package as
follows :

\begin{longtable}[]{@{}lll@{}}
\toprule
0 - 3 & 4 - 13 & 14 - 15\tabularnewline
\midrule
\endhead
M\_TYPE & EAT\_LOC & G\_NUM\tabularnewline
\bottomrule
\end{longtable}

\begin{itemize}
\item
  M\_TYPE : a 4 bit type field that specifies message types.
  \texttt{pacman\_ate\_ghost} has decimal value 7.
\item
  EAT\_LOC : a pair of coordinate X Y with two unsigned 5 bit integer to
  describe the ghost eating location of the AWAY pacman on the REMOTE
  maze.
\item
  G\_NUM : a unsigned 2 bit integer to describe the color of the ghost
  that is being eaten by the AWAY pacman on the REMOTE maze.
\end{itemize}

\begin{quote}
\begin{longtable}[]{@{}ll@{}}
\toprule
Decimal Value & Eaten Ghost Color\tabularnewline
\midrule
\endhead
0 & RED\tabularnewline
1 & BLUE\tabularnewline
2 & YELLOW\tabularnewline
3 & PINK\tabularnewline
\bottomrule
\end{longtable}
\end{quote}

\hypertarget{message-transmitted-using-udp}{%
\subsubsection{Message Transmitted using
UDP}\label{message-transmitted-using-udp}}

\hypertarget{pacman_update}{%
\paragraph{\texorpdfstring{\texttt{pacman\_update}}{pacman\_update}}\label{pacman_update}}

\texttt{pacman\_update} message consists of a 5 byte package as follows
:

\begin{longtable}[]{@{}llllll@{}}
\toprule
0 - 3 & 4 - 19 & 20 - 29 & 30 - 31 & 32 - 33 & 34 - 39\tabularnewline
\midrule
\endhead
M\_TYPE & SEQ & P\_LOC & P\_DIR & P\_SPD & UNUSED\tabularnewline
\bottomrule
\end{longtable}

\begin{itemize}
\item
  M\_TYPE : a 4 bit type field that specifies message types.
  \texttt{pacman\_update} has decimal value 8.
\item
  SEQ : a 16 bit unsigned integer that increases by one for every new
  UDP message sent. If it reaches 65535, it wraps back round to 0.
\item
  P\_LOC : a pair of coordinate X Y with two 5 bit unsigned integer to
  describe the location of the AWAY pacman.
\item
  P\_DIR : a 2 bit unsigned integer to describe the direction AWAY
  pacman currently faces.
\end{itemize}

\begin{quote}
\begin{longtable}[]{@{}ll@{}}
\toprule
Decimal Value & Pacman Direction\tabularnewline
\midrule
\endhead
0 & UP\tabularnewline
1 & LEFT\tabularnewline
2 & DOWN\tabularnewline
3 & RIGHT\tabularnewline
\bottomrule
\end{longtable}
\end{quote}

\begin{itemize}
\item
  P\_SPD : a 2 bit unsigned integer to describe the speed of the AWAY
  pacman.
\item
  UNUSED : a 4 bit field used for bit alignment.
\end{itemize}

\hypertarget{ghost_update}{%
\paragraph{\texorpdfstring{\texttt{ghost\_update}}{ghost\_update}}\label{ghost_update}}

\texttt{ghost\_update} message consists of a 10 byte package as follows
:

\begin{longtable}[]{@{}llllll@{}}
\toprule
0 - 3 & 4 - 19 & 20 - 21 & 22 & 23 - 78 & 79\tabularnewline
\midrule
\endhead
M\_TYPE & SEQ & G\_ALIVE & G\_MODE & G\_INFO (G\_LOC, G\_DIR, G\_SPD) &
UNUSED\tabularnewline
\bottomrule
\end{longtable}

\begin{itemize}
\item
  M\_TYPE : a 4 bit type field that specifies message types.
  \texttt{ghost\_update} has decimal value 9.
\item
  SEQ : a 16 bit unsigned integer that increases by one for every new
  UDP message sent. If it reaches 65535, it wraps back round to 0.
\item
  G\_ALIVE : a 2 bit unsigned integer that indicates the number of
  ghosts alive in frightened mode (dead ghosts will return to base and
  revive).
\end{itemize}

\begin{quote}
Note : If received G\_ALIVE equals or more than 4, CLIENT should send
\texttt{status\_update} with G\_STATUS of 3 to terminate the game and
check for errors.
\end{quote}

\begin{itemize}
\tightlist
\item
  G\_MODE : a 1 bit boolean value that indicates whether in frightened
  mode or chase mode.
\end{itemize}

\begin{quote}
\begin{longtable}[]{@{}ll@{}}
\toprule
Boolean Value & Status\tabularnewline
\midrule
\endhead
TRUE & FRIGHTEN\tabularnewline
FALSE & CHASE\tabularnewline
\bottomrule
\end{longtable}
\end{quote}

\begin{itemize}
\item
  G\_INFO : a 56 bit field which contains 4 tuples of a pair of X Y
  coordinates with two 5 bit unsigned integers indicating the location
  of a ghost, a 2 bit unsigned integer indicating the direction of the
  ghost and a 2 bit unsigned integer indicating the speed of the ghost.
  The corresponding tuple of each ghost is ordered in the same way as it
  is in \texttt{pacman\_ate\_ghost}. The G\_LOC is orderd in the same
  way as \texttt{pacman\_update}.
\item
  UNUSED : a 1 bit field used for bit alignment.
\end{itemize}

\hypertarget{eat}{%
\paragraph{\texorpdfstring{\texttt{eat}}{eat}}\label{eat}}

\texttt{eat} message consists of a 4 byte package as follows :

\begin{longtable}[]{@{}lllll@{}}
\toprule
0 - 3 & 4 - 19 & 20 - 29 & 30 & 31\tabularnewline
\midrule
\endhead
M\_TYPE & SEQ & F\_LOC & F\_FOREIGN & F\_POWER\tabularnewline
\bottomrule
\end{longtable}

\begin{itemize}
\item
  M\_TYPE : a 4 bit type field that specifies message types.
  \texttt{eat} has demcimal value 10.
\item
  SEQ : a 16 bit unsigned integer that increases by one for every new
  UDP message sent. If it reaches 65535, it wraps back round to 0.
\item
  F\_LOC : a pair of coordinate X Y with two 5 bit unsigned integers
  indicating the location of eaten food.
\item
  F\_FOREIGN : a 1 bit boolean value that indicates whether our LOCAL
  pacman is eating food on a REMOTE maze.
\end{itemize}

\begin{quote}
\begin{longtable}[]{@{}ll@{}}
\toprule
Boolean Value & Status\tabularnewline
\midrule
\endhead
TRUE & REMOTE\tabularnewline
FALSE & LOCAL\tabularnewline
\bottomrule
\end{longtable}
\end{quote}

\begin{itemize}
\tightlist
\item
  F\_POWER : a 1 bit boolean value that indicates whether the food ate
  was a powerpill.
\end{itemize}

\hypertarget{score_update}{%
\paragraph{\texorpdfstring{\texttt{score\_update}}{score\_update}}\label{score_update}}

\texttt{score\_update} message consists of a 7 byte package as follows :

\begin{longtable}[]{@{}llll@{}}
\toprule
0 - 3 & 4 - 19 & 20 - 51 & 52 - 55\tabularnewline
\midrule
\endhead
M\_TYPE & SEQ & SCORE\_INFO & UNUSED\tabularnewline
\bottomrule
\end{longtable}

\begin{itemize}
\item
  M\_TYPE : a 4 bit type field that specifies message types.
  \texttt{score\_update} has decimal value 11.
\item
  SEQ : a 16 bit unsigned integer that increases by one for every new
  UDP message sent. If it reaches 65535, it wraps back round to 0.
\item
  SCORE\_INFO : a 32 bit unsigned integer that indicates the score of
  the LOCAL pacman.
\item
  UNUSED : a 4 bit field used for bit alignment.
\end{itemize}

\hypertarget{message-sanity-check}{%
\subsection{Message Sanity Check}\label{message-sanity-check}}

For all locations (P\_LOC, G\_LOC, EAT\_LOC, DIE\_LOC .. ) in messages,
if any of the X Y coordinate is out of range {[}0 .. 24{]}, CLIENT
SHOULD terminate the game by sending a \texttt{status\_update} with
G\_STATUS of 3 to terminate the game.

For all messages, M\_TYPE MUST fall in range {[}0 .. 11{]}, otherwise
messages will be regarded illegal and discarded.

All UNUSED fields in messages MAY be filled with any value as the CLIENT
wishes.

\hypertarget{message-timing}{%
\subsection{Message Timing}\label{message-timing}}

For all messages using TCP and \texttt{eat}, the message is sent
whenever necessary.

For all messages using UDP except for \texttt{eat}, update messages are
sent at a time interval of 20ms when frame rate is higher than 50 per
second. Otherwise, update messages will be sent every frame.

\end{document}
